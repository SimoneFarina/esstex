\documentclass[11pt
  , a4paper
  , article
  , oneside
%  , twoside
%  , draft
]{memoir}

\usepackage{essdocs}
\usepackage[numbers]{natbib}

\begin{document}

%% ESS Document Description
%%
\essdocdesc{Engineering Manual}

%% ESS Document Number
%%
\essdocnum{ESS-XXXXXXXX}

%% Date
%%
\date{\today}

%% ESS Document Revision Number
%%
\essdocrev{0.9}

%% ESS Document State
%%
\essdocstate{Draft}

%% ESS Document Classification
%%
\essdocclass{ESS Use Only}

%% Document Title
%%
\title{ESS ICS Engineering Manual}
\subtitle{for MRF PCIe-EVR-300}
%% Document Author(s), if more than one author,
%% use \newline instead of \\ or \linebreak in order to seperate them
\author{Javier Cereijo Garcia \newline Jeong Han Lee }

%% Document Reviewer(s) if more than one reviewer,
%% use \newline instead of \\ or \linebreak in order to seperate them
\reviewer{Timo Korhonen (Chief Engineer) \newline Timo Korhonen (Chief Engineer)}

%% Document Owner(s) if more than one owner,
%% use \newline instead of \\ or \linebreak in order to seperate them
\owner{ICS}

%% Document Approver(s) if more than one approver,
%% use \newline instead of \\ or \linebreak in order to seperate them
\approver{ICS}


\esstitle
\newpage
\tableofcontents
\newpage


\begin{abstract}
This document explains the process followed for testing the PCIe-EVR-300 by Micro-Research Finland (MRF) at ESS using the EEE. This document can be used to test new equipment at in-kind partners facilities or as step-by-step instructions to set up a basic EVR.
\end{abstract}


%%% Actual Document Start at below

\chapter{TEST}


\section{Introduction}
The PCIe-EVR-300 is an event receiver (EVR) for the MRF timing system. Its role is receiving and decoding the events emited by an event generator (EVG) through an optical cable and producing outputs based on those events.\\
Due to the small size of the hardware, it only has a SFP module and a high density output connector to interface to an IFB-300, which has 16 universal outputs.\\
The PCIe-EVR-300 is used in EPICS using the mrfioc2 \cite{MRFIOC2} driver.\\

This document explains the process followed for testing the PCIe-EVR-300; it starts by describing the hardware list and setup for the tests and the test environment, followed by the startup configuration, the test goals, the tests done and their outputs. Finally a summary and conclusions are presented.


\section{Test bench description}
\subsection{Hardware list and specification}
The hardware used for testing the PCIe-EVR-300 is:
\begin{itemize}
\item MRF PCIe-EVR-300
	\begin{itemize}
	\item TAG-146
	\item Serial number K114029
	\end{itemize}
\item PC with free PCIe slot
	\begin{itemize}
	\item Kontron iPC in ESSIIP Control Rack in ICS lab (ics-essiip-01)
	\end{itemize}
\item MRF IFB-300 output module and cable
	\begin{itemize}
	\item TAG-336
	\item Serial number K472056
	\end{itemize}
\item MRF EVG
	\begin{itemize}
	\item cPCI-EVG-230
	\item TAG-26
	\item Serial number E283016
	\end{itemize}
\item MRF Fan-out
	\begin{itemize}
	\item cPCI-FOUT-12
	\item TAG-27
	\item Serial number H385013 
	\end{itemize}
\item Optical cables
\item Oscilloscope
\item LEMO cable and LEMO to BNC adaptor
\end{itemize}


\subsection{Hardware setup}
The phisical setup of the hardware is shown in figure \ref{fig:diagram}.
\begin{figure}[!ht]
  \centering
    \includegraphics[width=0.8\textwidth]{./pictures/pcie_network.eps}
  \caption{Hardware setup}
  \label{fig:diagram}
\end{figure}


\subsection{Test environment description}
\begin{itemize}
\item OS: CentOS Linux release 7.1.1503 (Core)
\item Kernel release: 3.10.0-229.7.2.rt56.141.6.el7.centos.x86\textunderscore 64
\item EEE version 1.8.0
\item EPICS base 3.14.12.5
\item mrfioc2 version 3.0.0
\item EVR FPGA FW version 0x17000008
\end{itemize}


\subsection{PCI information}
The output of \verb|lspci| for the PCIe-EVR-300 is:
\begin{lstlisting}
$ lspci -s 05:00 -vv
05:00.0 Signal processing controller: Lattice Semiconductor Corporation Device
        ec30 (rev 01)
    Subsystem: Device 1a3e:172c
    Control: I/O+ Mem+ BusMaster+ SpecCycle- MemWINV- VGASnoop- ParErr-
        Stepping- SERR- FastB2B- DisINTx-
    Status: Cap+ 66MHz- UDF- FastB2B- ParErr- DEVSEL=fast >TAbort- <TAbort-
        <MAbort- >SERR- <PERR- INTx-
    Latency: 0, Cache Line Size: 64 bytes
    Interrupt: pin A routed to IRQ 16
    Region 0: Memory at c0400000 (32-bit, non-prefetchable) [size=1M]
    Capabilities: <access denied>
    Kernel driver in use: mrf-pci

$ lspci -s 05:00 -t
-+-[0000:05]---00.0
 \-[0000:00]-
\end{lstlisting}

The PCI information available from EPICS is:

\begin{lstlisting}
ics-essiip-01> devPCIShow 9 0x1204
PCI 0000:05:00.0 IRQ 16
  vendor:device 1204:ec30 rev 00
  subved:subdev 1a3e:172c
  class 118000 generic signal processing controller
  driver mrf-pci
  BAR 0 32-bit MMIO      1 MB
Matched 1 devices
\end{lstlisting}


\subsection{Startup configuration}
The database used is evr-pcie-300.db and can be founf in the EEE in \newline
\texttt{/opt/epics/modules/mrfioc2/3.0.0/db/evr-pcie-300.db}.

The cPCI EVG is set to send event 14 at 14 Hz with an event clock frequency of 88.0525 MHz.
Since this frequency is different to the default in \texttt{evr-pcie-300.db}, the file is copied to the current folder and the link clock frequency is set to 88.0525 MHz (line 111): \texttt{field( VAL , "\$(Link-Clk-SP=88.0525,undefined)")}

MRF kernel module loaded with: \texttt{\$modprobe mrf}

Startup script:
\begin{lstlisting}
epicsEnvSet("SYS"             "SYS0")
epicsEnvSet("EVR"             "EVR0")
epicsEnvSet("EVR_PCIDOMAIN"   "0")
epicsEnvSet("EVR_PCIBUS"      "0x5")
epicsEnvSet("EVR_PCIDEVICE"   "0x0")
epicsEnvSet("EVR_PCIFUNCTION" "0")

require mrfioc2

mrmEvrSetupPCI($(EVR), $(EVR_PCIDOMAIN), $(EVR_PCIBUS), $(EVR_PCIDEVICE), $(EVR_PCIFUNCTION))

dbLoadRecords("./evr-pcie-300.db", "EVR=$(EVR), SYS=$(SYS)")
\end{lstlisting}

It is possible that the IOC registration function fails:
\begin{lstlisting}
mrmEvrSetupPCI(EVR0, 0, 0x5, 0x0, 0)
Device EVR0  5:0.0
Using IRQ 16
Can neither open resource file nor uio file of PCI device 0000:05:00.0 BAR 0
PCI error: Failed to map BARs 0 for EC 30
\end{lstlisting}
The solution can be found in the "Timing System: User Manual - Troubleshooting" wiki page \cite{TSESSWIKI}. Then the output should look like:
\begin{lstlisting}
$ iocsh test1.cmd 
/opt/epics/bases/base-3.14.12.5/bin/centos7-x86_64/softIoc -D
       /opt/epics/bases/base-3.14.12.5/dbd/softIoc.dbd /tmp/iocsh.startup.14439
#date="Wed  6 Jul 13:38:40 UTC 2016"
#user="javiercereijogarcia"
#PWD="/home/javiercereijogarcia/PCIe-EVR-300_tests"
#EPICSVERSION="3.14.12.5"
#EPICS_HOST_ARCH="centos7-x86_64"
#SHELLBOX=""
#EPICS_CA_ADDR_LIST=""
#EPICS_MODULE_INCLUDE_PATH=".:/usr/lib64:/usr/lib:/lib64:/lib"
dlload         /opt/epics/modules/environment/1.8.0/3.14.12.5/lib/centos7-x86_64/libenvironment.so
dbLoadDatabase /opt/epics/modules/environment/1.8.0/3.14.12.5/dbd/environment.dbd
environment_registerRecordDeviceDriver
< "test1.cmd"
epicsEnvSet("SYS"             "SYS0")
epicsEnvSet("EVR"             "EVR0")
epicsEnvSet("EVR_PCIDOMAIN"   "0")
epicsEnvSet("EVR_PCIBUS"      "0x5")
epicsEnvSet("EVR_PCIDEVICE"   "0x0")
epicsEnvSet("EVR_PCIFUNCTION" "0")
require mrfioc2
require: mrfioc2 depends on devlib2 (2.6+).
require: Loading library /opt/epics/modules/devlib2/2.6.0/3.14.12.5/lib/centos7-x86_64/libdevlib2.so.
require: Loading /opt/epics/modules/devlib2/2.6.0/3.14.12.5/dbd/devlib2.dbd.
require: Calling devlib2_registerRecordDeviceDriver function.
require: Loading library /opt/epics/modules/mrfioc2/3.0.0/3.14.12.5/lib/centos7-x86_64/libmrfioc2.so.
require: Adding /opt/epics/modules/mrfioc2/3.0.0/db.
require: Adding /opt/epics/modules/mrfioc2/3.0.0/startup.
require: Loading /opt/epics/modules/mrfioc2/3.0.0/3.14.12.5/dbd/mrfioc2.dbd.
require: Calling mrfioc2_registerRecordDeviceDriver function.
 
mrmEvrSetupPCI(EVR0, 0, 0x5, 0x0, 0)
Device EVR0  5:0.0
Using IRQ 16
Setting magic LE number!
EC 30: Enabling interrupts
FWVersion 0x17000008
Found version 8
Found EVR0:SFP0 EEPROM
PCIe: Out FP:0 FPUNIV:16 RB:0 IFP:0 GPIO:0
EVR FIFO task start
dbLoadRecords("./evr-pcie-300.db", "EVR=EVR0, SYS=SYS0")
iocInit
Starting iocInit
############################################################################
## EPICS R3.14.12.5-2015-08 $Date: Tue 2015-03-24 09:57:35 -0500$
## EPICS Base built Oct  9 2015
############################################################################
\end{lstlisting}


\section{PCIe-EVR-300 testing}
****************************************\\
The tests will have three parts: in the first one, the link between the EVG and the EVR is checked, and we check that the EVR is capable of synchronizing its frequency to the EVG; in the second test we check that the EVR is capable of receiving and acknowledging events; and finally we test the outputs.\\
****************************************\\

\subsection{Test 1 - checking the EVR-EVG link and setting event clock and link frequency}
\begin{lstlisting}
$ caget SYS0-EVR0:Link-Sts
SYS0-EVR0:Link-Sts             OK
  
$ caget SYS0-EVR0:Link-Clk-I
SYS0-EVR0:Link-Clk-I           88.0519
$ caput SYS0-EVR0:Link-Clk-SP 100
Old : SYS0-EVR0:Link-Clk-SP          88.0525
New : SYS0-EVR0:Link-Clk-SP          100
$ caget SYS0-EVR0:Link-Sts
SYS0-EVR0:Link-Sts             Fail
$ caget SYS0-EVR0:Link-Clk-I
SYS0-EVR0:Link-Clk-I           100
$ caput SYS0-EVR0:Link-Clk-SP 88.0525
Old : SYS0-EVR0:Link-Clk-SP          100
New : SYS0-EVR0:Link-Clk-SP          88.0525
$ caget SYS0-EVR0:Link-Sts
SYS0-EVR0:Link-Sts             OK
$ caget SYS0-EVR0:Link-Clk-I
SYS0-EVR0:Link-Clk-I           88.0519
$ camonitor SYS0-EVR0:Cnt-LinkTimo-I
SYS0-EVR0:Cnt-LinkTimo-I       2016-06-30 10:23:23.949149 18

## In another terminal:
$ caput SYS0-EVR0:Link-Clk-SP 100
Old : SYS0-EVR0:Link-Clk-SP          88.0525
New : SYS0-EVR0:Link-Clk-SP          100

## So the link timeout starts counting:
SYS0-EVR0:Cnt-LinkTimo-I       2016-06-30 10:24:12.037271 19
SYS0-EVR0:Cnt-LinkTimo-I       2016-06-30 10:24:13.735983 20
SYS0-EVR0:Cnt-LinkTimo-I       2016-06-30 10:24:15.438585 21
SYS0-EVR0:Cnt-LinkTimo-I       2016-06-30 10:24:17.137761 22
SYS0-EVR0:Cnt-LinkTimo-I       2016-06-30 10:24:18.836612 23

## And we stop it by:
$ caput SYS0-EVR0:Link-Clk-SP 88.0525
Old : SYS0-EVR0:Link-Clk-SP          100
New : SYS0-EVR0:Link-Clk-SP          88.0525
\end{lstlisting}

\subsection{Test 2 - monitoring event reception on the EVR}
For this test a line needs to be added to the startup script:
\begin{lstlisting}
dbLoadRecords("evr-softEvent.template", "EVR=$(EVR), SYS=$(SYS), EVT=14,
        CODE=14")

$ camonitor SYS0-EVR0:Event-14-Cnt-I
SYS0-EVR0:Event-14-Cnt-I       2016-06-30 13:55:36.008609 132
SYS0-EVR0:Event-14-Cnt-I       2016-06-30 13:55:36.080038 133
SYS0-EVR0:Event-14-Cnt-I       2016-06-30 13:55:36.151467 134
SYS0-EVR0:Event-14-Cnt-I       2016-06-30 13:55:36.222895 135
SYS0-EVR0:Event-14-Cnt-I       2016-06-30 13:55:36.294324 136
^C
\end{lstlisting}


\subsection{Test 3 - reacting to events by producing digital triggers}
For this test two lines need to be added to the startup script:
\begin{lstlisting}
dbLoadRecords("evr-pulserMap.template", "EVR=$(EVR), SYS=$(SYS), PID=0, F=Trig, ID=0, EVT=14")
dbLoadRecords("evr-pulserMap.template", "EVR=$(EVR), SYS=$(SYS), PID=1, F=Trig, ID=0, EVT=14")
\end{lstlisting}

PCIe-EVR universal output 0 and 1 are connected to channels 1 and 2 of the oscilloscope and:
\begin{lstlisting}
$ caput SYS0-EVR0:FrontUnivOut0-Src-SP 0
Old : SYS0-EVR0:FrontUnivOut0-Src-SP 63
New : SYS0-EVR0:FrontUnivOut0-Src-SP 0
$ caput SYS0-EVR0:Pul0-Evt-Trig0-SP      14
Old : SYS0-EVR0:Pul0-Evt-Trig0-SP    14
New : SYS0-EVR0:Pul0-Evt-Trig0-SP    14
$ caput SYS0-EVR0:Pul0-Width-SP     10000
Old : SYS0-EVR0:Pul0-Width-SP        0
New : SYS0-EVR0:Pul0-Width-SP        10000
\end{lstlisting}

The oscilloscope shows the signal shown in figure \ref{fig:14Hz}.
\begin{figure}[!ht]
  \centering
    \includegraphics[width=0.8\textwidth]{./pictures/img_1965.eps}
  \caption{14 Hz signal with 10 ms width}
  \label{fig:14Hz}
\end{figure}

\begin{lstlisting}
$ caput SYS0-EVR0:Pul0-Width-SP     50000
Old : SYS0-EVR0:Pul0-Width-SP        10000
New : SYS0-EVR0:Pul0-Width-SP        50000
\end{lstlisting}

The oscilloscope shows the signal shown in figure \ref{fig:50ms}.
\begin{figure}[!ht]
  \centering
    \includegraphics[width=0.8\textwidth]{./pictures/img_1966.eps}
  \caption{14 Hz signal with 50 ms width}
  \label{fig:50ms}
\end{figure}

\begin{lstlisting}
$ caput SYS0-EVR0:Pul0-Width-SP     40000
Old : SYS0-EVR0:Pul0-Width-SP        50000
New : SYS0-EVR0:Pul0-Width-SP        40000
$ caput SYS0-EVR0:Pul0-Width-SP     80000
Old : SYS0-EVR0:Pul0-Width-SP        40000
New : SYS0-EVR0:Pul0-Width-SP        80000
\end{lstlisting}

Produces in another terminal:
\begin{lstlisting}
$ camonitor SYS0-EVR0:Pul0-Width-RB
SYS0-EVR0:Pul0-Width-RB        2016-07-01 16:05:10.512348 50000
SYS0-EVR0:Pul0-Width-RB        2016-07-01 16:07:25.632305 40000
SYS0-EVR0:Pul0-Width-RB        2016-07-01 16:08:10.190883 80000
\end{lstlisting}

The delay tuning is also checked:
\begin{lstlisting}
$ caput SYS0-EVR0:Pul0-Width-SP     20000
Old : SYS0-EVR0:Pul0-Width-SP        80000
New : SYS0-EVR0:Pul0-Width-SP        20000
$ caput SYS0-EVR0:FrontUnivOut1-Src-SP 1
Old : SYS0-EVR0:FrontUnivOut1-Src-SP 63
New : SYS0-EVR0:FrontUnivOut1-Src-SP 0
$ caput SYS0-EVR0:Pul1-Evt-Trig0-SP      14
Old : SYS0-EVR0:Pul1-Evt-Trig0-SP    14
New : SYS0-EVR0:Pul1-Evt-Trig0-SP    14
$ caput SYS0-EVR0:Pul1-Width-SP     20000
Old : SYS0-EVR0:Pul1-Width-SP        0
New : SYS0-EVR0:Pul1-Width-SP        20000
$ caput SYS0-EVR0:Pul1-Delay-SP     30000
Old : SYS0-EVR0:Pul1-Delay-SP        0
New : SYS0-EVR0:Pul1-Delay-SP        30000
\end{lstlisting}

The oscilloscope shows figure \ref{fig:delay}.
\begin{figure}[!ht]
  \centering
    \includegraphics[width=0.8\textwidth]{./pictures/img_1982.eps}
  \caption{Two 14 Hz signals with 30 ms delay}
  \label{fig:delay}
\end{figure}

Finally all other outputs are checked in the same way.


\section{Conclusions}



%% \chapter{Requirements}

%% \section{FAI Server}
%% The most works will be done in so-called a faiserver, as a typical PC. We uses Debian 7 Wheezy up-to-date version as OS and its system information (\texttt{uname -a} is as follows:
%% \begin{lstlisting}
%% Linux faiserver 3.2.0-4-amd64 #1 SMP Debian 3.2.57-3+deb7u1 x86_64 GNU/Linux
%% \end{lstlisting}

%% \section{Media Access Control (MAC) addresses of Target Computers}
%% We need to get MAC address of target PCs. Here we use two PCs : one is the real PC, and another is the virtual one by vmplayer. The real one is named as \textit{demohost} and the virtual one does \textit{gnomehost}. Two names are simply used by using the fai-example configuration. So the total computers are \textbf{faiserver}, \textbf{demohost}, and \textbf{gnomehost}. Figure~\ref{fig:vmplayer_network} shows the vmplayer screen-shot of the MAC address for a virtual PC.

%% For the installation examples, two MAC addresses and the hostnames are used as follows :
%% \begin{itemize}
%% \item demohost  : \texttt{E8:03:9A:63:83:81}
%% \item gnomehost : \texttt{00:50:56:39:D9:1E}
%% \end{itemize}
%% \begin{figure}[!htb]
%%   \centering
%%   \includegraphics[width=0.96\textwidth]{./images/vmplayer_network.eps}
%%   \caption{
%%             MAC address setup for a virtual PC by using vmplayer.
%%           }
%%   \label{fig:vmplayer_network}   
%% \end{figure}

%% %%  Get MAC address on target PC (demohost, gnomehost)
%% %%   - VMWare Host : Network Bridget, Generate MAC address, use it in /etc/dhcp/dhcpd.conf
%% %%   - real   Host : Get MAC address...

%% \clearpage

%% \section{Packages}
%% We will use the up-to-date fai packages from fai-project site\footnote{\url{http://fai-project.org}} instead of the Debian Wheezy repository. The first thing is to modify the apt sources list to let the system know where the fai project is

%% \begin{lstlisting}%[caption={Modify the apt source list}]

%% root@faiserver:~# emacs /etc/apt/sources.list

%%   deb http://ftp.lecl.net/debian/ wheezy main contrib non-free
%%   deb http://fai-project.org/download wheezy koeln

%% root@faiserver:~# aptitude update

%% \end{lstlisting}

%% The next step is to install the fai packages such as \texttt{fai-client}, \texttt{fai-server}, and \texttt{fai\-quickstart}. Debian will care the other package dependencies, so simply use the following command : 

%% \begin{lstlisting}%[caption={Installation of \texttt{fai-client}, \texttt{fai-server}, and \texttt{fai-quickstart}}]

%% root@faiserver:/etc# aptitude install fai-client fai-server fai-quickstart

%% The following NEW packages will be installed:
%%   debootstrap{a} fai-client fai-doc{a} fai-quickstart fai-server libapt-pkg-perl{a} libgraph-perl{a} libheap-perl{a} 
%%   libproc-daemon-perl{a} libproc-processtable-perl{a} nfs-kernel-server{a} openbsd-inetd{a} reprepro{a} 
%% 0 packages upgraded, 13 newly installed, 0 to remove and 0 not upgraded.
%% Need to get 0 B/2,137 kB of archives. After unpacking 4,829 kB will be used.
%% Do you want to continue? [Y/n/?] 

%% ......................................
%% ......................................


%% Setting up nfs-kernel-server (1:1.2.6-4) ...
%% [ ok ] Stopping NFS kernel daemon: mountd nfsd.
%% [ ok ] Unexporting directories for NFS kernel daemon....
%% [warn] Not starting NFS kernel daemon: no exports. ... (warning).
%% Setting up openbsd-inetd (0.20091229-2) ...
%% [ ok ] Stopping internet superserver: inetd.
%% [info] Not starting internet superserver: no services enabled.
%% Setting up debootstrap (1.0.48+deb7u1) ...
%% Setting up fai-client (4.2) ...
%% Setting up fai-doc (4.2) ...
%% Setting up fai-server (4.2) ...
%% You might want to edit fai.conf and nfsroot.conf in /etc/fai or
%% go with the defaults. You should edit /etc/fai/apt/sources.list
%% for faster access to a package repository.
%% Setting up reprepro (4.12.5-1) ...
%% Setting up fai-quickstart (4.2) ...

%% \end{lstlisting}


%% \section{Dynamic Host Configuration Protocol (DHCP) Configuration}
%% The DHCP setting has all FAI host computers' MAC address and its ip address. And if there is no isc-dhcp packages, please install them by using the following command:
%% \begin{lstlisting}
%% root@faiserver:~# aptitude install isc-dhcp-server
%% \end{lstlisting}
%% Since the DHCP configuration is essential for the FAI setup, please edit dhcp daemon configuration file (dhcpd.conf) as follows:

%% \begin{lstlisting}
%% root@faiserver:/etc# emacs /etc/dhcp/dhcpd.conf
%% \end{lstlisting}
%% The following example is used in the \texttt{10.1.4.0} subnet inside the RISP 2nd floor office network. So, the ip range is limited for only two addresses (210 and 211). The \texttt{domain-name} is provided by the local nameserver (\texttt{10.1.5.11}). The gateway of the \texttt{10.1.4.0} subnet is used as \texttt{routers}. The DHCP server has \texttt{10.1.4.100}.

%% \begin{lstlisting}
%% ddns-update-style none;
%% log-facility local7;

%% deny unknown-clients;
%% option dhcp-max-message-size 2048;
%% use-host-decl-names on;

%% subnet 10.1.4.0 netmask 255.255.255.0 {

%%        option domain-name         "risp.invalid";
%%        option domain-name-servers 10.1.5.11;
%%        option routers             10.1.4.254;

%%        range  10.1.4.210 10.1.4.211;
%%        default-lease-time 60;
%%        max-lease-time 720;

%%        next-server 10.1.4.100;
%%        filename "fai/pxelinux.0";
%% }

%% host demohost {
%%         hardware ethernet e8:03:9a:63:83:81;
%%         fixed-address 10.1.4.210;
%% }

%% host gnomehost{
%%         hardware ethernet 00:50:56:39:D9:1E;
%%         fixed-address 10.1.4.211;
%% }
%% \end{lstlisting}
%% In addition, we have one more example which is a local \textit{home} domain, because of the better understanding for the DHCP configuration. In this example, the router is the DHCP server which provides an automatic IP address to wire and wireless clients. The faiserver (\texttt{10.0.185.12}) is one of the wire client and the target, i.e., a host with \texttt{10.0.185.200}, is the virtual image on the faiserver. 

%% \begin{lstlisting}

%% dns-update-style none;

%% default-lease-time 600;
%% max-lease-time 7200;

%% log-facility local7;


%% deny unknown-clients;
%% option dhcp-max-message-size 2048;
%% use-host-decl-names on;

%% subnet 10.0.185.0 netmask 255.255.255.0 {
%%        # network settings
%%        option domain-name         "home.invalid";
%%        option domain-name-servers 10.0.185.1;
%%        option routers             10.0.185.1;

%%        # client IP allocation
%%        range  10.0.185.200 10.0.185.210;
%%        default-lease-time 60;
%%        max-lease-time 720;

%%        # PXE boot server
%%        next-server 10.0.185.12;
%%        filename "fai/pxelinux.0";

%% }

%% host demohost {
%%         hardware ethernet 00:50:56:36:52:2A;
%%         fixed-address 10.0.185.200;
%% }
%% \end{lstlisting}

%% After saving the configuration, one should restart or start the DHCP server on the faiserver as follows:     
%% \begin{lstlisting}
%% root@faiserver:/etc# /etc/init.d/isc-dhcp-server restart
%% [FAIL] Stopping ISC DHCP server: dhcpd failed!
%% [ ok ] Starting ISC DHCP server: dhcpd.
%% \end{lstlisting}
%% It the DHCP server never be executed before, Stoping process will be failed as the above what we see. 


%% \section{The \texttt{/etc/hosts} file} 

%% One should add the FAI server and FAI hosts IP addresses and their hostname in the /etc/hosts file. The information must be the same as what is inside /etc/dhcp/dhcpd.conf file. 
%% %% #
%% %% # Edit /etc/hosts file in order to add the FAI server and FAI hosts
%% %% #
%% \begin{lstlisting}
%% 127.0.0.1       localhost
%% 10.1.4.100      faiserver
%% 10.1.4.210      demohost
%% 10.1.4.211      gnomehost

%% \end{lstlisting}


%% \clearpage

%% \chapter{FAI Settings}
%% In this chapter, we configure many FAI settings one by one. There are two important directories : 1) the FAI server configuration directory (\texttt{/ect/fai}) is created by the fai packages, and 2) another one (\texttt{/srv/fai}) is created by the \texttt{fai-setup} command as follows :
%% \begin{itemize}
%%   \item \texttt{/etc/fai}
%%     {\scriptsize
%%      \begin{verbatim}
%%       root@faiserver:/etc/fai# tree
%%       .
%%       ├── apt
%%       │   └── sources.list
%%       ├── fai.conf
%%       ├── grub.cfg
%%       ├── live.conf
%%       ├── NFSROOT
%%       └── nfsroot.conf
%%      \end{verbatim}
%%      }
%%   \item \texttt{/srv/fai}
%%     {\scriptsize
%%      \begin{verbatim}
%%       root@faiserver:/srv# tree -d -L 3
%%       .
%%       ├── fai
%%       │   ├── config
%%       │   │   ├── basefiles
%%       │   │   ├── class
%%       │   │   ├── debconf
%%       │   │   ├── disk_config
%%       │   │   ├── files
%%       │   │   ├── hooks
%%       │   │   ├── package_config
%%       │   │   ├── scripts
%%       │   │   └── tests
%%       │   └── nfsroot
%%       │       ├── bin
%%       │       ├── boot
%%       │       ├── dev
%%       │       ├── etc
%%       │       ├── home
%%       │       ├── lib
%%       │       ├── lib64
%%       │       ├── media
%%       │       ├── mnt
%%       │       ├── opt
%%       │       ├── proc
%%       │       ├── root
%%       │       ├── run
%%       │       ├── sbin
%%       │       ├── selinux
%%       │       ├── srv
%%       │       ├── sys
%%       │       ├── tmp
%%       │       ├── usr
%%       │       └── var
%%       └── tftp
%%             └── fai
%%                   └── pxelinux.cfg
%%   \end{verbatim}
%%   }
%% \end{itemize}


%% \section{The \texttt{/etc/fai/apt/sources.list} File}
%% In the same way, we should change the Debian repository with the local (S.Korea) one, i.e., \texttt{http://ftp.lecl.net/debian/}, in order to speed up our installation method. 
%% Please edit \texttt{/etc/fai/apt/sources.list} as follows:
%% \begin{lstlisting}
%% root@faiserver:/etc/fai/apt# emacs sources.list 

%%     #deb http://http.debian.net/debian wheezy main contrib non-free
%%     deb http://ftp.lecl.net/debian/ wheezy main contrib non-free
%%     deb http://security.debian.org/debian-security wheezy/updates main contrib non-free

%%     # repository that may contain newer fai packages for wheezy
%%     deb http://fai-project.org/download wheezy koeln

%% \end{lstlisting}

%% \section{NFSROOT Configuration}

%% \subsection{GPG Key for NFSROOT}
%% %% # Prepare not to meet the GPG error: http://fai-project.org wheezy after adding fai-project.org in the nfsroot repository
%% The GPG key of the fai project repository must be installed in nfsroot. Note that the \texttt{chroot /srv/fai/nfsroot/} command must be before \texttt{apt-key add -}.
%% \begin{lstlisting}
%% root@faiserver:/etc/fai/apt# wget -O - http://fai-project.org/download/074BCDE4.asc | chroot /srv/fai/nfsroot/ apt-key add -

%% --2014-06-05 14:01:15--  http://fai-project.org/download/074BCDE4.asc
%% Resolving fai-project.org (fai-project.org)... 134.95.9.240
%% Connecting to fai-project.org (fai-project.org)|134.95.9.240|:80... connected.
%% HTTP request sent, awaiting response... 200 OK
%% Length: 5607 (5.5K) [text/plain]
%% Saving to: `STDOUT'

%% 100\%[============================================>] 5,607       --.-K/s   in 0s      
%% 2014-06-05 14:01:16 (762 MB/s) - written to stdout [5607/5607]
%% OK
%% \end{lstlisting}

%% \subsection{Debian Repository for NFSROOT}
%% We change the Debian repository as the local one  to save the installation or downloading time, e.g., \texttt{http://ftp.lecl.net/debian}. 

%% % # change FAI_DEBOOTSTRAP as the local one in order to reduce download time

%% \begin{lstlisting}
%% root@faiserver:/etc/fai/apt# emacs /etc/fai/nfsroot.conf 

%%   # For a detailed description see nfsroot.conf(5)

%%   # "<suite> <mirror>" for debootstrap
%%   FAI_DEBOOTSTRAP="wheezy http://ftp.lecl.net/debian/"
%%   FAI_ROOTPW='\$1\$kBnWcO.E\$djxB128U7dMkrltJHPf6d1'

%%   NFSROOT=/srv/fai/nfsroot
%%   TFTPROOT=/srv/tftp/fai
%%   NFSROOT_HOOKS=/etc/fai/nfsroot-hooks/
%%   FAI_DEBOOTSTRAP_OPTS="--exclude=info"

%%   # Configuration space
%%   FAI_CONFIGDIR=/srv/fai/config
%% \end{lstlisting}


%% \subsection{Package List for NFSROOT}
%% %% #
%% %% # Remove the console-tools and console-common
%% %% # Replace ntpdate with ntp
%% %% # Add firmware-linux-nonfree
%% %% # Add console-setup-linux
%% One must change the default package list from the FAI packages, because there are some dependent issues, which are described in Troubleshooting later. Here the summary what I've modified as follows:
%% \begin{itemize}
%%   \item \textbf{mandatory :} remove the troublesome \texttt{console-tools} and \texttt{console-common} packages in the line number \ref{troublesome_console} in Listing~\ref{list:nfsroot-file}.
%%   \item \textbf{mandatory :} add \texttt{console-setup-linux} and \texttt{console-data} in the line number \ref{console-setup-linux} in Listing~\ref{list:nfsroot-file} in order to compensate two troublesome \texttt{console-tools} and \texttt{console-common} packages.
%%   \item \textbf{convenience :} replace \texttt{ntpdate} with \texttt{ntp} in the line numbers \ref{ntpdate} and \ref{ntp} in Listing~\ref{list:nfsroot-file}
%%   \item \textbf{mandatory for some FAI hosts :} add \texttt{firmware-linux-nonfree} for selected demohost. If not, after successfully installation, there is no network connection on the demohost in the line number \ref{firmware-linux-nonfree} in Listing~\ref{list:nfsroot-file}. 
%% \end{itemize}

%% \begin{lstlisting}[style=termstylenumber, caption={Editing \texttt{/etc/fai/NFSROOT}}, label={list:nfsroot-file}]
%% root@faiserver:/etc/fai# emacs -nw NFSROOT 

%%    # package list for creating the NFSROOT

%%    PACKAGES aptitude
%%    nfs-common fai-nfsroot module-init-tools ssh rdate lshw rpcbind
%%    rsync lftp less dump reiserfsprogs e2fsprogs usbutils
%%    hwinfo psmisc pciutils hdparm smartmontools parted mdadm lvm2
%%    dnsutils
%%    #ntpdate  (*@\label{ntpdate}@*) 
%%    ntp       (*@\label{ntp}@*) 
%%    dosfstools xfsprogs xfsdump
%%    procinfo numactl dialog
%%    #console-tools console-common (*@\label{troublesome_console}@*) 
%%    console-setup-linux console-data           (*@\label{console-setup-linux}@*) 
%%    kbd
%%    iproute moreutils udev subversion
%%    xz-utils
%%    cupt

%%    # some network cards needs firmware
%%    firmware-bnx2 firmware-bnx2x firmware-realtek
%%    firmware-linux-nonfree        (*@\label{firmware-linux-nonfree}@*) 

%%    # dracut can replace live-boot
%%    dracut-network live-boot- live-boot-initramfs-tools-

%%    # choose if you like live-boot or dracut inside the nfsroot
%%    #live-boot live-boot-doc

%%    # you should not edit the lines below
%%    # architecture dependend list of packages that are installed

%%    #git # git consumes a lot of disk space on the FAI CD (ISO 9660)

%%    PACKAGES aptitude I386
%%    grub-pc
%%    linux-image-686

%%    # packages for Ubuntu natty/oneiric/precise:
%%    # linux-image-generic live-boot

%%    PACKAGES aptitude AMD64
%%    grub-pc
%%    linux-image-amd64

%%    # packages for Ubuntu natty/oneiric/precise:
%%    # linux-image-generic live-boot
%% \end{lstlisting}


%% \section{FAI Setup}

%% Now, ready to execute \texttt{fai-setup} command. With \texttt{-v} option, we can see the verbose output.
%% \subsection{First \texttt{fai-setup} command}

%% \begin{lstlisting}[style=termstylenumber, caption={fai-setup output}, label={list:fai-setup-output}]
%% root@faiserver:/etc/fai# fai-setup -v
%% Using configuration files from /etc/fai
%% Creating FAI nfsroot in /srv/fai/nfsroot
%% Creating base system using debootstrap version 1.0.48+deb7u1
%% Calling debootstrap --exclude=info wheezy /srv/fai/nfsroot http://ftp.lecl.net/debian/
%% I: Retrieving Release
%% I: Retrieving Release.gpg
%% I: Checking Release signature
%% .....................

%% WARNING: untrusted versions of the following packages will be installed!

%% Untrusted packages could compromise your system's security.
%% You should only proceed with the installation if you are certain that
%% this is what you want to do.

%%   liblinux-lvm-perl fai-nfsroot fai-client fai-setup-storage 

%% *** WARNING ***   Ignoring these trust violations because
%%                   aptitude::CmdLine::Ignore-Trust-Violations is 'true'!
%% Writing extended state information...
%% ...................
%% `/srv/fai/nfsroot/boot/vmlinuz-3.2.0-4-amd64' -> `/srv/tftp/fai/vmlinuz-3.2.0-4-amd64'
%% `/srv/fai/nfsroot/boot/initrd.img-3.2.0-4-amd64' -> `/srv/tftp/fai/initrd.img-3.2.0-4-amd64'
%% TFTP environment prepared. Enable DHCP and start the TFTP daemon on root /srv/tftp/fai.
%% FAI packages inside the nfsroot:
%% fai-client         4.2
%% fai-nfsroot        4.2
%% fai-setup-storage  4.2
%% FAI related packages inside the nfsroot:
%% dracut             037-1
%% dracut-network     037-1
%% Waiting for background jobs to finish
%% [1]+  Done                    nice xz -q \$NFSROOT/var/tmp/base.tar  (wd: /srv/fai/nfsroot)
%% fai-make-nfsroot finished properly.
%% Log file written to /var/log/fai/fai-make-nfsroot.log
%% Adding line to /etc/exports: /srv/fai/nfsroot 10.1.4.100/24(async,ro,no_subtree_check,no_root_squash) (*@\label{exports}@*) 
%% Re-exporting directories for NFS kernel daemon....

%%    You have no FAI configuration space yet. Copy the simple examples with:
%%    cp -a /usr/share/doc/fai-doc/examples/simple/* /srv/fai/config (*@\label{configuration_example}@*) 
%%    Then change the configuration files to meet your local needs.
%% Please don't forget to fill out the FAI questionnaire after you've finished your project with FAI.

%% FAI setup finished.
%% Log file written to /var/log/fai/fai-setup.log
%% \end{lstlisting}

%% \subsection{NFS setup}
%% The \texttt{fai-setup} command adds one important nfs path in \texttt{/etc/exports} in the line number \ref{exports} in Listing~\ref{list:fai-setup-output}. However, it is necessary to add another nfs path in \texttt{/etc/exports}. Consult the line number \ref{fai-config} in Listing~\ref{list:etc-exports}.
%% \begin{lstlisting}[style=termstylenumber, caption={Add \texttt{/srv/fai/config} directory to \texttt{/etc/exports}}, label={list:etc-exports}]
%% root@faiserver:/etc/fai# emacs /etc/exports

%%   /srv/fai/nfsroot 10.1.4.100/24(async,ro,no_subtree_check,no_root_squash)
%%   /srv/fai/config  10.1.4.100/24(async,ro,no_subtree_check) (*@\label{fai-config}@*) 
%% \end{lstlisting}
%% After that, restart NFS services 

%% \begin{lstlisting}
%% root@faiserver:/etc/fai#  /etc/init.d/nfs-kernel-server restart

%% [ ok ] Stopping NFS kernel daemon: mountd nfsd.
%% [ ok ] Unexporting directories for NFS kernel daemon....
%% [ ok ] Exporting directories for NFS kernel daemon....
%% [ ok ] Starting NFS kernel daemon: nfsd mountd.
%% \end{lstlisting}
%% %% root@faiserver:/etc/fai# 

%% \subsection{Example Configurations : demohost and gnomehost}
%% \begin{itemize}
%% \item According to the line number \ref{configuration_example} in Listing~\ref{list:etc-exports}, we simply copy the example configuration into \texttt{/srv/fai/config}.
%% \begin{lstlisting}
%% root@faiserver:/etc/fai# cp -a /usr/share/doc/fai-doc/examples/simple/* /srv/fai/config
%% \end{lstlisting}

%% \item Generate boot parameters for two example hosts which are named as \texttt{demohost} and \texttt{gnomehost}.
%% %% #
%% %% # Generate a fai-client configuration for demohost / gnomehost
%% %% #

%% \begin{lstlisting}[style=termstylenumber, caption={Generate Boot Parameters for \texttt{demohost} and \texttt{gnomehost}}, label={list:boot-parameters}]
%% root@faiserver:/etc/fai#  fai-chboot -IFv -u nfs://10.1.4.100/srv/fai/config demohost

%% Booting kernel vmlinuz-3.2.0-4-amd64
%%  append initrd=initrd.img-3.2.0-4-amd64 ip=dhcp  
%%    FAI_FLAGS=verbose,sshd,createvt FAI_CONFIG_SRC=nfs://10.1.4.100/srv/fai/config

%% demohost has 10.1.4.210 in hex 0A0104D2
%% Writing file /srv/tftp/fai/pxelinux.cfg/0A0104D2 for demohost (*@\label{demohost-boot-parameter}@*) 

%% root@faiserver:/etc/fai#  fai-chboot -IFv -u nfs://10.1.4.100/srv/fai/config gnomehost

%% Booting kernel vmlinuz-3.2.0-4-amd64
%%  append initrd=initrd.img-3.2.0-4-amd64 ip=dhcp  
%%    FAI_FLAGS=verbose,sshd,createvt FAI_CONFIG_SRC=nfs://10.1.4.100/srv/fai/config

%% gnomehost has 10.1.4.211 in hex 0A0104D3
%% Writing file /srv/tftp/fai/pxelinux.cfg/0A0104D3 for gnomehost (*@\label{gnomehost-boot-parameter}@*) 
%% \end{lstlisting}

%% \item Edit the boot parameters in the line numbers \ref{demohost-boot-parameter} and \ref{gnomehost-boot-parameter} in Listing~\ref{list:boot-parameters}. If not, we will meet the message like as \texttt{can't open /etc/fstab}. This is the bug of NFS.V4, so this change makes to let systems to use NFS.V3. 
%% \begin{lstlisting}[emph={nfsroot}, emphstyle=\underbar, style=termstylenumber, caption={Default Boot Parameters}, label={list:default-boot-parameter}]
%% root@faiserver:/etc/fai# emacs /srv/tftp/fai/pxelinux.cfg/0A0104D2

%%   # generated by fai-chboot for host demohost with IP 10.1.4.210
%%   default fai-generated

%%   label fai-generated
%%   kernel vmlinuz-3.2.0-4-amd64
%%   append initrd=initrd.img-3.2.0-4-amd64 ip=dhcp  root=/srv/fai/nfsroot aufs  FAI_FLAGS=verbose,sshd,createvt FAI_CONFIG_SRC=nfs://10.1.4.100/srv/fai/config FAI_ACTION=install (*@\label{nfsroot_v3}@*) 
%% \end{lstlisting}
%% Add the \texttt{:vers=3} at the end of \texttt{root=/srv/fai/nfsroot} in the line number~\ref{nfsroot_v3} in Listing~\ref{list:default-boot-parameter}. So it should be 

%% \begin{lstlisting}[emph={nfsroot, vers}, emphstyle=\underbar]
%%   root=/srv/fai/nfsroot:vers=3 
%% \end{lstlisting}
%% Modify \texttt{/srv/tftp/fai/pxelinux.cfg/0A0104D3} also. 

%% \item Restart tftpd-hpa (Trivial File Transfer Protocol Server) service. 

%% \begin{lstlisting}
%% root@faiserver:/etc/fai# /etc/init.d/tftpd-hpa restart

%% [ ok ] Restarting HPA's tftpd: in.tftpd.
%% \end{lstlisting}

%% \item Add FAI repository GPG key to \texttt{/srv/fai/nfsroot}

%% \begin{lstlisting}
%% root@faiserver:/etc/fai# wget -O - http://fai-project.org/download/074BCDE4.asc | chroot /srv/fai/nfsroot/ apt-key add -
%% --2014-06-05 14:21:43--  http://fai-project.org/download/074BCDE4.asc
%% Resolving fai-project.org (fai-project.org)... 134.95.9.240
%% Connecting to fai-project.org (fai-project.org)|134.95.9.240|:80... connected.
%% HTTP request sent, awaiting response... 200 OK
%% Length: 5607 (5.5K) [text/plain]
%% Saving to: `STDOUT'
%% 100\%[==========================================>] 5,607       --.-K/s   in 0s      
%% 2014-06-05 14:21:44 (749 MB/s) - written to stdout [5607/5607]
%% OK
%% \end{lstlisting}
%% \end{itemize}

%% \section{Start gnomehost}
%% Figure~\ref{fig:gnomehost} shows the whole screen shots for the gnomehost, which is the virtual PC by vmplayer.

%% \begin{figure}[!htb]
%%   \centering
 
%%   \subbottom[F12 lets us to boot this host]
%% 	    {
%% 	      \includegraphics[width=0.45\textwidth]{./images/f-0.eps}
%% 	      \label{fig:f-0}
%% 	    }
%%             \hfill
%%   \subbottom[Booting .... ]
%% 	    {
%% 	      \includegraphics[width=0.45\textwidth]{./images/f-1.eps}
%% 	      \label{fig:f-1}
%% 	    }
%%             \subbottom[Clearly see the FAI Screen]
%% 	    {
%% 	      \includegraphics[width=0.45\textwidth]{./images/f-2.eps}
%% 	      \label{fig:f-2}
%% 	    }
%%             \hfill
%%   \subbottom[After reboot, the grub screen is shown]
%% 	    {
%% 	      \includegraphics[width=0.45\textwidth]{./images/f-3.eps}
%% 	      \label{fig:f-3}
%% 	    }
%%             \subbottom[GNOME login screen]
%% 	    {
%% 	      \includegraphics[width=0.45\textwidth]{./images/f-4.eps}
%% 	      \label{fig:f-4}
%% 	    }
%%             \hfill
%%   \subbottom[IP address in Gnome terminal]
%% 	    {
%% 	      \includegraphics[width=0.45\textwidth]{./images/f-5.eps}
%% 	      \label{fig:f-5}
%% 	    }
%%   \caption
%%       {
%%         FAI Installation screenshots and the installed Debian Linux
%%       }
%%  \label{fig:gnomehost}
%% \end{figure}


%% \section{Update FAI Setup}
%% Technically, it is unnecessary to update the FAI setup. Once one follows the instructions one by one, it should work. I think, from this ugly installation experience, it may be necessary to run later on after changing any default packages and so on. However, I am not sure when we need this later on. It should be tested a bit more later. 

%% \begin{lstlisting}
%% root@faiserver:/etc/fai# fai-setup -vk
%% Using configuration files from /etc/fai
%% Upgrading nfsroot and installing new packages into the nfsroot.
%% Hit http://ftp.lecl.net wheezy Release.gpg

%% ........................
%% ........................

%% install_packages: executing chroot /srv/fai/nfsroot apt-get clean
%% install_packages: executing chroot /srv/fai/nfsroot dpkg --configure --pending
%% install_packages: executing chroot /srv/fai/nfsroot dpkg -C
%% install_packages: executing chroot /srv/fai/nfsroot apt-get clean
%% install_packages exit code: 0
%% `/srv/fai/nfsroot/boot/vmlinuz-3.2.0-4-amd64' -> `/srv/tftp/fai/vmlinuz-3.2.0-4-amd64'
%% `/srv/fai/nfsroot/boot/initrd.img-3.2.0-4-amd64' -> `/srv/tftp/fai/initrd.img-3.2.0-4-amd64'
%% TFTP environment prepared. Enable DHCP and start the TFTP daemon on root /srv/tftp/fai.
%% fai-make-nfsroot finished properly.W: GPG error: http://fai-project.org wheezy Release: 
%% The following signatures couldn't be verified because the public key is not available: NO_PUBKEY 2BF8D9FE074BCDE4

%% Re-exporting directories for NFS kernel daemon....
%% FAI setup finished.
%% Log file written to /var/log/fai/fai-setup.log

%% \end{lstlisting}

%% %% # The demohost has demo or root users and with 'fai' password
%% \chapter{Build i386 Compatible FAI Server on x86\_{64}}
%% There are two references \citep{faii386,faiguide} to this issue. Both methods do not give me the full instruction for setup to build i386 compatible FAI server. So, this chapter was developed by a traditional trial and error. However, they \citep{faii386,faiguide} gave me a way where I should go. The general procedure is almost the same as the normal procedure, which is described before. Therefore, I use the same way to do so here. 

%% \section{Requirements}
%% I slightly modify what I've done before to rename gnomehost to ctrldemo01, and add a virtual i386 PC in a vmplayer. So ctrldemo01 is the 64bit virtual PC. Note that one must select a proper virtual image setting for an i386 PC in a vmplayer.
%% \subsection{MAC addresses of target computers}
%% \begin{itemize}
%%   \item demohost   : \texttt{e8:03:9a:63:83:81}, a PC
%%   \item ctrldemo01 : \texttt{00:50:56:39:D9:1E}, a virtual 64bit PC
%%   \item ctrldemo02 : \texttt{00:50:56:3E:0B:CE}, a virtual i386 PC
%% \end{itemize}
%% \subsection{DHCP configuration}
%% The existing DHCP configuration is modified as follows :
%% \begin{lstlisting}
%% ddns-update-style none;

%% log-facility local7;
 
%% deny unknown-clients;
%% option dhcp-max-message-size 2048;
%% use-host-decl-names on;

%% subnet 10.1.4.0 netmask 255.255.255.0 {
%%        # network settings
%%        option domain-name         "risp.invalid";
%%        option domain-name-servers 10.1.5.11;
%%        option routers             10.1.4.254;

%%        # client IP allocation
%%        range  10.1.4.210 10.1.4.219;
%%        default-lease-time 60;
%%        max-lease-time 720;

%%        # PXE boot server
%%        next-server 10.1.4.100;
%%        filename "fai/pxelinux.0";

%% }


%% host demohost {
%%         hardware ethernet e8:03:9a:63:83:81;
%%         fixed-address 10.1.4.210;
%% }


%% host ctrldemo01{
%%         hardware ethernet 00:50:56:39:D9:1E;
%%         fixed-address 10.1.4.211;
%% }

%% host ctrldemo02 {
%%         hardware ethernet 00:50:56:3E:0B:CE;
%%         fixed-address 10.1.4.212;
%% }

%% \end{lstlisting}
%% After saving the configuration, one should restart or start the DHCP server on the faiserver as follows:     
%% \begin{lstlisting}
%% root@faiserver:/etc/dhcp# /etc/init.d/isc-dhcp-server restart
%% [ ok ] Stopping ISC DHCP server: dhcpd.
%% [ ok ] Starting ISC DHCP server: dhcpd.
%% root@faiserver:/etc/dhcp# 
%% \end{lstlisting}


%% \subsection{The \texttt{/etc/hosts} file}
%% One should add the FAI hosts' IP addresses and their hostnames in the \texttt{/etc/hosts} file as follows:

%% \begin{lstlisting}
%% 127.0.0.1       localhost
%% 10.1.4.100      faiserver
%% 10.1.4.210      demohost
%% 10.1.4.211      ctrldemo01
%% 10.1.4.212      ctrldemo02
%% 10.1.4.213      ctrldemo03
%% 10.1.4.214      ctrldemo04
%% 10.1.4.215      ctrldemo05
%% 10.1.4.216      ctrldemo06
%% 10.1.4.217      ctrldemo07
%% 10.1.4.218      ctrldemo08
%% 10.1.4.219      ctrldemo09
%% \end{lstlisting}
%% And the list is generated by a perl command in the reference \citep{faiguide} Chapter 9 as follows :
%% \begin{lstlisting}
%% perl -e 'for (1..9) {printf "10.1.4.21%s ctrldemo%02s\n",$_,$_;}'

%% 10.1.4.211 ctrldemo01
%% 10.1.4.212 ctrldemo02
%% 10.1.4.213 ctrldemo03
%% 10.1.4.214 ctrldemo04
%% 10.1.4.215 ctrldemo05
%% 10.1.4.216 ctrldemo06
%% 10.1.4.217 ctrldemo07
%% 10.1.4.218 ctrldemo08
%% 10.1.4.219 ctrldemo09
%% \end{lstlisting}
%% I add seven more ctrldemo for further tests.

%% \section{i386 FAI Settings}
%% In this section, we re-use the existent fai setting of a 64bit. Thus, first of all, simple copy them to a new directory which name is \texttt{fai-i386} as 

%% \begin{lstlisting}
%% root@faiserver:/etc# cp -a /etc/fai /etc/fai-i386
%% \end{lstlisting}

%% \subsection{Create New Host Classes for ctrldemo hosts}

%% The ctrldemo host names are not in the class in \texttt{/srv/fai/config}, so I add it to be the same as \texttt{gnomehost} as 

%% \begin{lstlisting}
%% root@faiserver:/srv/fai/config/class# emacs 50-host-classes 
%% # use a list of classes for our demo machine                                                                                                                
%% case \$HOSTNAME in
%%     ctrldemo*)
%%         echo "FAIBASE DEBIAN DHCPC DEMO XORG GNOME";;
%% \end{lstlisting}


%% \subsection{NFSROOT configuration for i386}

%% \begin{itemize}

%% \item GPG Key

%% \begin{lstlisting}
%% root@faiserver:/etc/fai-i386# wget -O - http://fai-project.org/download/074BCDE4.asc | chroot /srv/fai/nfsroot-i386/ apt-key add -
%% --2014-07-07 19:44:54--  http://fai-project.org/download/074BCDE4.asc
%% Resolving fai-project.org (fai-project.org)... 134.95.9.240
%% Connecting to fai-project.org (fai-project.org)|134.95.9.240|:80... connected.
%% HTTP request sent, awaiting response... 200 OK
%% Length: 5607 (5.5K) [text/plain]
%% Saving to: `STDOUT'

%% 100\%[====================================================>] 5,607       --.-K/s   in 0s      

%% 2014-07-07 19:44:55 (487 MB/s) - written to stdout [5607/5607]

%% OK
%% \end{lstlisting}

%% \item Edit \texttt{nfsroot.conf} file for i386 hosts. I add the \texttt{i386} suffix in the line numbers \ref{i386-1} and include the architecture in \texttt{FAI\_DEBOOTSTRAP\_OPTS} in the line number \ref{i386-2} in Listing~\ref{list:nfsroot-conf-i386-file}.

%% \begin{lstlisting}[style=termstylenumber, caption={Editing \texttt{nfsroot.conf} for i386}, label={list:nfsroot-conf-i386-file}]
%% # For a detailed description see nfsroot.conf(5)

%% # "<suite> <mirror>" for debootstrap
%% FAI_DEBOOTSTRAP="wheezy http://ftp.lecl.net/debian/"
%% FAI_ROOTPW='\$1\$kBnWcO.E\$djxB128U7dMkrltJHPf6d1'

%% NFSROOT=/srv/fai/nfsroot-i386 (*@\label{i386-1}@*) 
%% TFTPROOT=/srv/tftp/fai
%% NFSROOT_HOOKS=/etc/fai/nfsroot-hooks/
%% FAI_DEBOOTSTRAP_OPTS="--arch i386 --exclude=info" (*@\label{i386-2}@*) 

%% # Configuration space
%% FAI_CONFIGDIR=/srv/fai/config
%% \end{lstlisting}


%% \end{itemize}


%% \section{i386 FAI Setup}
%% \subsection{First \texttt{fai-setup} command}
%% \begin{lstlisting}
%% root@faiserver:/srv/fai/config/class# fai-setup -C /etc/fai-i386 -vk
%% \end{lstlisting}
%% , where the options are 
%% \begin{itemize}
%% \item \texttt{-C} : Use \texttt{CFDIR} as the configuration directory. Default is \texttt{/etc/fai}. You can also set the environment variable \texttt{FAI\_ETC\_DIR}.
%% \item \texttt{-k} : Update/install packages defined in NFSROOT config.
%% \item \texttt{-v} : I guess this option is Verbose mode, because I cannot find it in manpage and so on. 
%% \end{itemize}

%% \subsection{Check NFS environment}
%% Check the NFS environment for i386 in \texttt{/etc/exports} which should have the following line :
%% \begin{lstlisting}
%% /srv/fai/nfsroot-i386 10.1.4.100/24(async,ro,no_subtree_check,no_root_squash)
%% \end{lstlisting}

%% After that, restart NFS services (is it necessary?)

%% \begin{lstlisting}
%% root@faiserver:/etc/fai#  /etc/init.d/nfs-kernel-server restart

%% [ ok ] Stopping NFS kernel daemon: mountd nfsd.
%% [ ok ] Unexporting directories for NFS kernel daemon....
%% [ ok ] Exporting directories for NFS kernel daemon....
%% [ ok ] Starting NFS kernel daemon: nfsd mountd.
%% \end{lstlisting}

%% \subsection{Create Boot Parameters for \texttt{ctrldemo02} host}
%% First of all, we must select proper Linux Kernel Executable (vmlinuz) and Initial RAM disk (initrd) for i386 in \texttt{/srv/tftp/fai/} directory.

%% \begin{lstlisting}
%% root@faiserver:/srv/fai# ls /srv/tftp/fai/initrd.img-* 
%% /srv/tftp/fai/initrd.img-3.2.0-4-686-pae  /srv/tftp/fai/initrd.img-3.2.0-4-amd64
%% root@faiserver:/srv/fai# ls /srv/tftp/fai/vmlinuz-3.2.0-4-* 
%% /srv/tftp/fai/vmlinuz-3.2.0-4-686-pae  /srv/tftp/fai/vmlinuz-3.2.0-4-amd64
%% \end{lstlisting}
%% By using \texttt{fai-chboot} command, 
%% \begin{lstlisting}
%% root@faiserver:/etc/fai-i386# fai-chboot -IFv -s3.2.0-4-686-pae -u nfs://10.1.4.100/srv/fai/config -C /etc/fai-i386/ ctrldemo02
%% Booting kernel vmlinuz-3.2.0-4-686-pae
%%  append initrd=initrd.img-3.2.0-4-686-pae ip=dhcp  
%%    FAI_FLAGS=verbose,sshd,createvt FAI_CONFIG_SRC=nfs://10.1.4.100/srv/fai/config

%% ctrldemo02 has 10.1.4.212 in hex 0A0104D4
%% Writing file /srv/tftp/fai/pxelinux.cfg/0A0104D4 for ctrldemo02
%% root@faiserver:/etc/fai-i386# fai-chboot -L
%% ctrldemo02 0A0104D4 vmlinuz-3.2.0-4-686-pae initrd=initrd.img-3.2.0-4-686-pae ip=dhcp  root=/srv/fai/nfsroot-i386 aufs  FAI_FLAGS=verbose,sshd,createvt FAI_CONFIG_SRC=nfs://10.1.4.100/srv/fai/config FAI_ACTION=install
%% \end{lstlisting}
%% , where the options are 
%% \begin{itemize}
%% \item \texttt{-s} : Use SUFFIX to determine which kernel and initrd to use.
%% \item \texttt{-u} : Set \texttt{FAI\_CONFIG\_SRC} to \texttt{URL}. Setting this variable is mandatory for the operation of FAI.
%% \item \texttt{-C} : Use \texttt{CFDIR} as the configuration directory. Default is \texttt{/etc/fai}. You can also set the environment variable \texttt{FAI\_ETC\_DIR}.
%% \item \texttt{-I} : Same as -i but also sets \texttt{FAI\_ACTION=install}. So a fully automatic installation will be performed. ATTENTION! This will erase most of the data  on  the local disks of the install clients.
%% \item \texttt{-F} : Set default values for \texttt{FAI\_FLAGS}. This is the same as -f verbose,sshd,createvt
%% \end{itemize}
%% And after adding \texttt{:vers=3}, 
%% \begin{lstlisting}
%% root@faiserver:/etc/fai-i386# fai-chboot -L
%% ctrldemo02 0A0104D4 vmlinuz-3.2.0-4-686-pae initrd=initrd.img-3.2.0-4-686-pae ip=dhcp  root=/srv/fai/nfsroot-i386:vers=3 aufs  FAI_FLAGS=verbose,sshd,createvt FAI_CONFIG_SRC=nfs://10.1.4.100/srv/fai/config FAI_ACTION=install
%% \end{lstlisting}
%% And restart tftpd-hpa (Trivial File Transfer Protocol Server) service. 
%% \begin{lstlisting}
%% root@faiserver:/etc/fai-i386# /etc/init.d/tftpd-hpa restart
%% [ ok ] Restarting HPA's tftpd: in.tftpd.
%% \end{lstlisting}

%% And turn on \texttt{ctrldemo01} and \texttt{ctrldemo02} in a vmplayer, all FAI installation are done well. See the Figure~\ref{fig:ctrldemo1and2} for its results.

%% \begin{figure}[!htb]
%%   \centering
 
%%   \subbottom[\texttt{ctrldemo01} : x86\_{64} host]
%% 	    {
%% 	      \includegraphics[width=0.45\textwidth]{./images/ctrldemo1_amd64.eps}
%% 	      \label{fig:ctrldemo1_amd64}
%% 	    }
%%             \hfill
%%   \subbottom[\texttt{ctrldemo02} : x86 host]
%% 	    {
%% 	      \includegraphics[width=0.45\textwidth]{./images/ctrldemo2_i386.eps}
%% 	      \label{fig:ctrldemo2_i386}
%% 	    }
%%   \caption
%%       {
%%         i386 and x86\_{64} compatible FAI server on x86\_{64}
%%       }
%%  \label{fig:ctrldemo1and2}
%% \end{figure}

%% %\newpage
%% \chapter{Troubleshooting}
%% This chapter has trial and error while installing and configuring the FAI server. 

%% \section{Kernel Panic}
%% One can see the following message during a host network booting :

%% \begin{lstlisting}
%% can't open /etc/fstab.... >>> kernel panic
%% \end{lstlisting}
%% As far as I understand, NFS.V4 doesn't fully support some features of FAI, so force to use V3. One can add \texttt{:vers=3} in the generated network boot configuration. For example, the generated network boot configuration of the demohost is located in 
%% \begin{lstlisting}
%% /srv/tftp/fai/pxelinux.cfg/0A0104D2
%% \end{lstlisting}

%% %% ** Case 2 **
%% \section{GPG Error}
%% Debian has been using strong crypto to validate downloaded packages \citep{DEBIANSecureApt}. Each release files has a GPG signature. If the system, i.e., faiserver, does not know the GPG signature of fai-project, the following error is shown as
%% \begin{lstlisting}
%% W: GPG error: http://fai-project.org wheezy Release: 
%% The following signatures couldn't be verified because the public key is not available: NO_PUBKEY 2BF8D9FE074BCDE4
%% \end{lstlisting}
%% The easy way to resolve to let the system (faiserver) knows the gpg key of the fai-project. 

%% \begin{lstlisting}
%% root@faiserver:/etc/fai# wget -O - http://fai-project.org/download/074BCDE4.asc | chroot /srv/fai/nfsroot/ apt-key add -
%% --2014-06-05 14:21:43--  http://fai-project.org/download/074BCDE4.asc
%% Resolving fai-project.org (fai-project.org)... 134.95.9.240
%% Connecting to fai-project.org (fai-project.org)|134.95.9.240|:80... connected.
%% HTTP request sent, awaiting response... 200 OK
%% Length: 5607 (5.5K) [text/plain]
%% Saving to: `STDOUT'

%% 100\%[=============================================>] 5,607       --.-K/s   in 0s      

%% 2014-06-05 14:21:44 (749 MB/s) - written to stdout [5607/5607]

%% OK

%% \end{lstlisting}

%% \section{Deprecated Option}
%% If one uses only the Debian Wheezy repositories, one should see the following lines in the network boot configuration, e.g., \texttt{/srv/tftp/fai/pxelinux.cfg/0A0104D2}, generated by fai-chroot as 
%% \begin{lstlisting}
%% append initrd=initrd.img-3.2.0-4-amd64 ip=dhcp  root=/dev/nfs nfsroot=/srv/fai/nfsroot aufs  
%% FAI_FLAGS=verbose,sshd,createvt FAI_CONFIG_SRC=nfs://10.1.4.100/srv/fai/config FAI_ACTION=install
%% \end{lstlisting}
%% instead of 
%% \begin{lstlisting}
%% append initrd=initrd.img-3.2.0-4-amd64 ip=dhcp  root=/srv/fai/nfsroot aufs  
%% FAI_FLAGS=verbose,sshd,createvt FAI_CONFIG_SRC=nfs://10.1.4.100/srv/fai/config FAI_ACTION=install
%% \end{lstlisting}

%% The option \texttt{root=/dev/nfs nfsroot=/srv/fai/nfsroot} is depreciated \citep{dracut}. The manpage of the dracut.cmdline \citep{dracutmanpage} shows 
%% \begin{lstlisting}
%% root=/dev/nfs nfsroot=[<server-ip>:]<root-dir>[:<nfs-options>]
%%            Deprecated!  kernel Documentation_/filesystems/nfsroot.txt_
%%            defines this method. This is supported by dracut, but not
%%            recommended.
%% \end{lstlisting}
%% The recent dracut is provided by the fai-project repository. Therefore, it is the better way to use \textbf{one} repository selected from Debian Wheezy or fai project. Both cases works well with the additional :vers=3 option.


%% %% ** Case 4 **
%% \section{Debian Package Dependency Error}
%% If one follow the instruction before, the following error will not be shown. However, just in case that one might see the following error :
%% \begin{lstlisting}

%% Errors were encountered while processing:
%%  linux-image-3.2.0-4-amd64
%% E: Sub-process /usr/bin/dpkg returned an error code (1)
%% A package failed to install.  Trying to recover:
%% ...........................
%% 1 errors during executing of commands
%% \end{lstlisting}
%% If one sees the error message, \texttt{ chroot /srv/fai/nfsroot/ apt-get install -f} may solve this problem temporarily. However, it will fail again later on. This error, as far as I know, is related the following section problem.

%% \section{Infinite Error Messages with fai-setup -vk command}
%% %% ** Case 5 **
%% One might see the following infinite error messages when fai-setup -vk is issued in both cases : 1) only Debian Wheezy repository and 2) Debian Wheezy/fai-project repositories. 

%% \begin{lstlisting}

%% root@faiserver:/srv/fai/nfsroot/etc/apt# fai-setup -vk

%% ............................................

%% Building dependency tree...
%% Reading state information...
%% The following NEW packages will be installed:
%%   ca-certificates curl libcurl3 librtmp0 libssh2-1 openssl
%% The following packages will be upgraded:
%%   dracut dracut-network fai-client fai-nfsroot fai-setup-storage
%%   liblinux-lvm-perl
%% 6 upgraded, 6 newly installed, 0 to remove and 0 not upgraded.
%% Need to get 2323 kB of archives.
%% After this operation, 3316 kB of additional disk space will be used.
%% WARNING: The following packages cannot be authenticated!
%%   librtmp0 libssh2-1 libcurl3 openssl ca-certificates curl dracut
%%   dracut-network liblinux-lvm-perl fai-client fai-setup-storage fai-nfsroot
%% Authentication warning overridden.

%% ............................................

%% The following NEW packages will be installed:
%%   console-tools{b} libconsole{a} 
%% 0 packages upgraded, 2 newly installed, 0 to remove and 0 not upgraded.
%% Need to get 467 kB of archives. After unpacking 1561 kB will be used.
%% The following packages have unmet dependencies:
%%  kbd : Conflicts: console-utilities which is a virtual package.
%%  console-tools : Conflicts: console-utilities which is a virtual package.
%% The following actions will resolve these dependencies:

%%      Remove the following packages:
%% 1)     dracut                      
%% 2)     dracut-network              
%% 3)     kbd                         
%% 4)     linux-image-3.2.0-4-amd64   
%% 5)     linux-image-amd64           

%% The following NEW packages will be installed:
%%   console-tools libconsole{a} 
%% The following packages will be REMOVED:
%%   dracut{a} dracut-network{a} kbd{a} linux-image-3.2.0-4-amd64{a} 
%%   linux-image-amd64{a} 
%% 0 packages upgraded, 2 newly installed, 5 to remove and 0 not upgraded.
%% Need to get 467 kB of archives. After unpacking 109 MB will be freed.
%% Writing extended state information...
%% Get: 1 http://ftp.lecl.net/debian/ wheezy/main libconsole amd64 1:0.2.3dbs-70 [155 kB]
%% Get: 2 http://ftp.lecl.net/debian/ wheezy/main console-tools amd64 1:0.2.3dbs-70 [312 kB]
%% Fetched 467 kB in 0s (5577 kB/s)
%% Can not write log, openpty() failed (/dev/pts not mounted?)
%% (Reading database ... 18224 files and directories currently installed.)
%% Removing dracut-network ...
%% Removing linux-image-amd64 ...
%% Removing linux-image-3.2.0-4-amd64 ...
%% Aborting removal of running kernel image.
%% dpkg: error processing linux-image-3.2.0-4-amd64 (--remove):
%%  subprocess installed pre-removal script returned error exit status 1
%% dpkg: dracut: dependency problems, but removing anyway as you requested:
%%  linux-image-3.2.0-4-amd64 depends on initramfs-tools (>= 0.99~) | linux-initramfs-tool; however:
%%   Package initramfs-tools is not installed.
%%   Package linux-initramfs-tool is not installed.
%%   Package dracut which provides linux-initramfs-tool is to be removed.

%% Removing dracut ...
%% Processing triggers for man-db ...
%% Errors were encountered while processing:
%%  linux-image-3.2.0-4-amd64
%% E: Sub-process /usr/bin/dpkg returned an error code (1)
%% A package failed to install.  Trying to recover:
%% Reading package lists...
%% Building dependency tree...
%% Reading state information...
%% Reading extended state information...
%% Initializing package states...
%% Writing extended state information...
%% Reading task descriptions...
%% ERROR: 65280 65280
%% ERROR: chroot /srv/fai/nfsroot aptitude -R -y -o Dpkg::Options::=--force-confdef -o Dpkg::Options::=--force-confnew install nfs-common fai-nfsroot module-init-tools ssh rdate lshw rpcbind rsync lftp less dump reiserfsprogs e2fsprogs usbutils hwinfo psmisc pciutils hdparm smartmontools parted mdadm lvm2 dnsutils ntpdate dosfstools xfsprogs xfsdump procinfo numactl dialog console-tools console-common kbd iproute moreutils udev subversion xz-utils cupt firmware-bnx2 firmware-bnx2x firmware-realtek dracut-network live-boot- live-boot-initramfs-tools- grub-pc linux-image-amd64 return code 255
%% install_packages: executing chroot /srv/fai/nfsroot apt-get clean
%% install_packages: executing chroot /srv/fai/nfsroot dpkg --configure --pending
%% install_packages: executing chroot /srv/fai/nfsroot dpkg -C
%% install_packages: executing chroot /srv/fai/nfsroot apt-get clean
%% 1 errors during executing of commands
%% Log file written to /var/log/fai/fai-setup.log


%% root@faiserver:/srv/fai/nfsroot/etc/apt# chroot /srv/fai/nfsroot/ apt-get install -f
%% Reading package lists... Done
%% Building dependency tree       
%% Reading state information... Done
%% Correcting dependencies... Done
%% The following extra packages will be installed:
%%   busybox initramfs-tools klibc-utils libklibc
%% Suggested packages:
%%   bash-completion
%% The following NEW packages will be installed:
%%   busybox initramfs-tools klibc-utils libklibc
%% 0 upgraded, 4 newly installed, 0 to remove and 0 not upgraded.
%% Need to get 792 kB of archives.
%% After this operation, 1661 kB of additional disk space will be used.
%% WARNING: The following packages cannot be authenticated!
%%   busybox libklibc klibc-utils initramfs-tools
%% Authentication warning overridden.

%% ...................

%% Fetched 792 kB in 1s (539 kB/s)          
%% perl: warning: Setting locale failed.
%% perl: warning: Please check that your locale settings:
%%         LANGUAGE = "en_US:en",
%%         LC_ALL = (unset),
%%         LANG = "en_US.UTF-8"
%%     are supported and installed on your system.
%% perl: warning: Falling back to the standard locale ("C").
%% locale: Cannot set LC_CTYPE to default locale: No such file or directory
%% locale: Cannot set LC_MESSAGES to default locale: No such file or directory
%% locale: Cannot set LC_ALL to default locale: No such file or directory
%% Can not write log, openpty() failed (/dev/pts not mounted?)
%% Selecting previously unselected package busybox.
%% (Reading database ... 17844 files and directories currently installed.)
%% Unpacking busybox (from .../busybox_1\%3a1.20.0-7_amd64.deb) ...
%% Selecting previously unselected package libklibc.
%% Unpacking libklibc (from .../libklibc_2.0.1-3.1_amd64.deb) ...
%% Selecting previously unselected package klibc-utils.
%% Unpacking klibc-utils (from .../klibc-utils_2.0.1-3.1_amd64.deb) ...
%% Selecting previously unselected package initramfs-tools.
%% Unpacking initramfs-tools (from .../initramfs-tools_0.109.1_all.deb) ...
%% Processing triggers for man-db ...
%% locale: Cannot set LC_CTYPE to default locale: No such file or directory
%% locale: Cannot set LC_MESSAGES to default locale: No such file or directory
%% locale: Cannot set LC_ALL to default locale: No such file or directory
%% Can not write log, openpty() failed (/dev/pts not mounted?)
%% Setting up busybox (1:1.20.0-7) ...
%% Setting up libklibc (2.0.1-3.1) ...
%% Setting up klibc-utils (2.0.1-3.1) ...
%% Setting up initramfs-tools (0.109.1) ...
%% update-initramfs: deferring update (trigger activated)
%% Processing triggers for initramfs-tools ...
%% update-initramfs: /boot/initrd.img-3.2.0-4-amd64 has been altered.
%% update-initramfs: Cannot update. Override with -t option.

%% root@faiserver:/srv/fai/nfsroot/etc/apt# chroot /srv/fai/nfsroot/ apt-get install bash-completion
%% Reading package lists... Done
%% Building dependency tree       
%% Reading state information... Done
%% The following NEW packages will be installed:
%%   bash-completion
%% 0 upgraded, 1 newly installed, 0 to remove and 0 not upgraded.
%% Need to get 193 kB of archives.
%% After this operation, 784 kB of additional disk space will be used.
%% WARNING: The following packages cannot be authenticated!
%%   bash-completion
%% Authentication warning overridden.
%% Get:1 http://ftp.lecl.net/debian/ wheezy/main bash-completion all 1:2.0-1 [193 kB]
%% Fetched 193 kB in 0s (2922 kB/s)        
%% perl: warning: Setting locale failed.
%% perl: warning: Please check that your locale settings:
%%         LANGUAGE = "en_US:en",
%%         LC_ALL = (unset),
%%         LANG = "en_US.UTF-8"
%%     are supported and installed on your system.
%% perl: warning: Falling back to the standard locale ("C").
%% locale: Cannot set LC_CTYPE to default locale: No such file or directory
%% locale: Cannot set LC_MESSAGES to default locale: No such file or directory
%% locale: Cannot set LC_ALL to default locale: No such file or directory
%% Can not write log, openpty() failed (/dev/pts not mounted?)
%% Selecting previously unselected package bash-completion.
%% (Reading database ... 17976 files and directories currently installed.)
%% Unpacking bash-completion (from .../bash-completion_1\%3a2.0-1_all.deb) ...
%% Processing triggers for man-db ...
%% locale: Cannot set LC_CTYPE to default locale: No such file or directory
%% locale: Cannot set LC_MESSAGES to default locale: No such file or directory
%% locale: Cannot set LC_ALL to default locale: No such file or directory
%% Can not write log, openpty() failed (/dev/pts not mounted?)
%% Setting up bash-completion (1:2.0-1) ...
%% \end{lstlisting}

%% It looks like that works again. However, it doesn't work. See Figure~\ref{fig:fai-error} for the possible error which one may see.

%% \begin{figure}[!htb]
%%   \centering
%%   \includegraphics[width=0.96\textwidth]{./images/fai-error.eps}
%%   \caption{
%%             A Booting Failure for a demohost within a virtual environment.
%%           }
%%   \label{fig:fai-error}   
%% \end{figure}

%% In order to solve this issue, it is necessary to remove two packages in \texttt{/etc/fai/NFSROOT} as follows:
%% \texttt{console-tools} and \texttt{console-common}. The exact reason why it is so, I don't know. However, if I remove two packages in \texttt{NFSROOT} and add \texttt{console-setup-linux} and \texttt{console-data} instead, the infinite install/uninstall procedures are disappeared.

%% %## 
%% %## Installation documents guided me a lot of failures, but the following wiki
%% %## did many helps : http://wiki.fai-project.org/wiki/Installation_walkthrough
%% %##

%% %Installation documents guided me a lot of failures, but the following wiki did many helps %%: http://wiki.fai-project.org/wiki/Installation_walkthrough
 
%% %% Useful package : fai-doc
%% %%  Check /usr/share/doc/fai-doc/examples/etc/

%% \section{Useful information}
%% \begin{itemize}
%%   \item One subject of the FAI project wiki \citep{FAIWikiWorkthrough}, gave me many helps to resolve some issues. 
%%   \item The useful package is fai-doc, will be installed at 
%%     \begin{lstlisting}
%%       /usr/share/doc/fai-doc/examples/etc/    \end{lstlisting}
%% \end{itemize}

\clearpage
%\bibliographystyle{unsrt}
%\bibliographystyle{plainnat}
%\bibliographystyle{abbrvnat}
\bibliographystyle{unsrtnat}
%\bibliographystyle{chicago}
\bibliography{./ess_refs}




\end{document}
